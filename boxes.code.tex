% boxes.code
%! suppress = GatherEquations

% Variables
\NewDocumentCommand{\boxing@Colums}{}{}
\NewDocumentCommand{\boxing@EqualHeight}{}{}
\NewDocumentCommand{\boxing@Palette}{}{}%
\newif\ifboxing@Continued


\NewDocumentCommand{\boxingset}{ >{\TrimSpaces} m }{%
    \pgfkeys{
        /boxing/.cd,
        #1
    }
}

% Default settings
\boxingset{
    columns = auto,
    equal height = all,
    row skip = 2ex,
    continued = false,
    palette = Toasted Peach,
}


% Tikz
\RequirePackage{tikz}
\tikzset{
    picture box/.style = {
        baseline = (text.base),
    },
    every text node/.style = {
        draw = \CurrentBGColor!70!black, ultra thick,
        ultra thick,
        text = \CurrentFGColor,
        font = \footnotesize,
        text width = \BoxWidth,
        inner xsep = 0.75em,
        inner ysep = 0.5em,
        shade,
        right color = \CurrentBGColor!70,
        left color = \CurrentBGColor,
        % bottom color = \CurrentBGColor!50,
    },
}

\NewDocumentCommand{\BoxBody}{}{}
\NewDocumentCommand{\HBox}{m}{%
    \UsePalette{\boxing@Palette}%
    \RenewDocumentCommand{\BoxBody}{}{\trim@spaces{#1}}%
    \begingroup%
    %! formatter = off
    \ifboxing@UseBeamer%
    \setbeamerfont{itemize/enumerate body}{size = \footnotesize}%
    \setbeamercolor{itemize/enumerate body}{fg = \CurrentFGColor}%
    \setbeamercolor{structure}{fg = \CurrentFGColor}%
    \fi%
    %! formatter = on
    \begin{tikzpicture}[picture box]
        \node[every text node] (text) {\raggedright\BoxBody\strut};
    \end{tikzpicture}%
    \endgroup%
    \StepPalette{\boxing@Palette}%
    \ifnum\thecurrentitemnumber=\thenumgathereditems\else%
    \hfill%
    \fi%
}

% Environments
\NewEnviron{horizontalbox}[1][]{%
    \expandafter\gatheritems\expandafter{\BODY}%
    \ResetPalette{\boxing@Palette}%
    \setlength{\BoxWidth}{
        \dimexpr \textwidth/\thenumgathereditems - 1.5em -
        \BoxHSpacing
    }%
    \loopthroughitemswithcommand{\HBox}%
}

\RequirePackage{ragged2e}

\RequirePackage{tcolorbox}
\newif\ifUseIcon
\UseIconfalse
\NewDocumentCommand{\TCBBoxCircle}{ m m +m }{%
    \visible<#1>{%
        \begin{tcolorbox}[
        left = 0.25em,
        right = 0.25em,
        bottom = 0pt,
        coltext = \CurrentFGColor,
        colback = \CurrentBGColor,
        colframe = \CurrentBGColor!70!black,
        toprule = 0.1em,
        bottomrule = 0.1em,
        leftrule = 0.1em,
        rightrule = 0.1em,
        enhanced,
        enlarge left by = 1em,
        enlarge top by = 1em,
        interior style = {
            top color = \CurrentBGColor,
            bottom color = \CurrentBGColor!80!white,
        },
        overlay = {
            \coordinate (North West) at ($(frame.north west) + (0.05em, -0.05em)$);
            \path [fill = \CurrentBGColor!90!black] (North West) circle (1em);
            \path [thick, draw = \CurrentBGColor!30!white] (North West) circle (0.9em);
            \path [draw = \CurrentBGColor!70!black, line width = 0.1em, ] ($(frame.north west) + (1.15em, -0.05em)$) -- ++(-0.1em, 0) arc (0:270:1em) -- ++(0, -0.1em);
            \node[font = \itshape, align = center, text = \CurrentFGColor] at ($(frame.north west) + (0.05em, -0.05em)$) {\ifUseIcon\includegraphics[width = 1.2em, height = 1.2em]{icons/#2}\else\scalebox{1.2}{#2}\fi};
        },
        ]
        \RaggedRight\rule{0pt}{2ex}#3
        \end{tcolorbox}%
    }%
}

\NewDocumentCommand{\TCBBoxPlain}{ m m +m }{%
    \visible<#1>{%
        \begin{tcolorbox}[
            left = 0.25em,
            right = 0.25em,
            bottom = 0pt,
            top = 0pt,
            coltext = \CurrentFGColor,
            colback = \CurrentBGColor,
            colframe = \CurrentBGColor!70!black,
            toprule = 0.1em,
            bottomrule = 0.1em,
            leftrule = 0.1em,
            rightrule = 0.1em,
            enhanced,
            interior style = {
                top color = \CurrentBGColor,
                bottom color = \CurrentBGColor!80!white,
            },
        ]
        \RaggedRight#3
        \end{tcolorbox}%
    }%
}


\NewDocumentCommand{\TCBInternal}{ m +m }{%
    \UsePaletteByNumber{\boxing@Palette}{\theCurrentItemNumber}%
    \ProcessItemArguments{\l@overlay}{\l@option}{\l@text}{#2}%
    %! formatter = off
    \ifx\l@overlay\empty%
        \RenewExpandableDocumentCommand{\l@overlay}{}{1-}%
    \fi
    \ifx\l@option\empty%
        \UseIconfalse%
        \RenewExpandableDocumentCommand{\l@option}{}{\theCurrentItemNumber}%
    \else
        \UseIcontrue%
    \fi%
    \ifboxing@UseBeamer%
        \setbeamerfont{itemize/enumerate body}{size = \footnotesize}%
        \setbeamercolor{itemize/enumerate body}{fg = \CurrentFGColor}%
        \setbeamercolor{structure}{fg = \CurrentFGColor}%
    \fi%
    %! formatter = on
    #1{\l@overlay}{\l@option}{\l@text}%
    \stepcounter{CurrentItemNumber}%
}

% interface commands
\NewDocumentCommand{\TCB}{ +m }{%
    \TCBInternal{\TCBBoxCircle}{#1}%
}

\NewDocumentCommand{\TCBsimple}{ +m }{%
    \TCBInternal{\TCBBoxPlain}{#1}%
}

\newcounter{CurrentItemNumber}
\NewDocumentEnvironment{mybox}{ O{} +b }{%
    \begingroup%
    \gatheritems{#2}%
    %! formatter = off
    \boxingset{continued = false}
    \pgfkeys{%
        /boxing/.cd,
        #1
    }%
    \ifboxing@Continued\else
        \ResetPalette{Colored Coolers}%
        \setcounter{CurrentItemNumber}{1}%
    \fi
    %! formatter = on
    \begin{tcbraster}[
        raster columns = \boxing@Colums,
        raster halign = center,
        raster right skip = \dimexpr 1em * \kvtcb@raster@columns \relax,
        nobeforeafter,
        raster column skip = 0.5em,
        /utils/exec = {%
        \pgfkeysalsofrom{/tcb/raster equal height = \boxing@EqualHeight}%
        },
    ]
        \loopthroughitemswithcommand{\TCB}%
        }{%
    \end{tcbraster}%
    \endgroup%
}

\NewDocumentEnvironment{plainbox}{ O{} +b }{%
    \begingroup%
    \gatheritems{#2}%
    %! formatter = off
    \boxingset{continued = false}
    \pgfkeys{%
        /boxing/.cd,
        #1
    }%
    \ifboxing@Continued\else
    \ResetPalette{Colored Coolers}%
    \setcounter{CurrentItemNumber}{1}%
    \fi
    %! formatter = on
    \begin{tcbraster}[
        raster columns = \boxing@Colums,
        raster halign = center,
        raster right skip = \dimexpr 1em * \kvtcb@raster@columns \relax,
        nobeforeafter,
        raster column skip = 0.5em,
        raster row skip = \boxing@RowSkip,% \dimexpr \boxing@RowSkip - 12pt \relax,
        raster equal height = \boxing@EqualHeight,
%        /utils/exec = {%
%        \pgfkeysalsofrom{/tcb/raster equal height = \boxing@EqualHeight}%
%        },
    ]
        \loopthroughitemswithcommand{\TCBsimple}%
        }{%
    \end{tcbraster}%
    \endgroup%
}


\endinput