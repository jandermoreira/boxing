%%%%%%%%%%%%%%%%%%%%%%%%%%%%%%%%%%%%%%%%%%%%%%%%%%%
%! Autor = Jander Moreira
%! Data = 10/10/2022

\documentclass[10pt]{beamer}
\usepackage[T1]{fontenc}
\usepackage[utf8]{inputenc}
\usepackage[brazilian]{babel}

\usetheme{Monjolinho}

\title{Title}
\subtitle{Subtitle}
\author{Author}

\usepackage[beamer]{boxing}

\usepackage[sfdefault]{FiraSans} %% option 'sfdefault' activates Fira Sans as the default text font
\usepackage[T1]{fontenc}
% \renewcommand*\oldstylenums[1]{{\firaoldstyle #1}}


\begin{document}

\begin{frame}
    Things that work

    \bigskip
    \hrulefill
    \begin{mybox}
        % \item \textbf{Algoritmos}\newline
        % Especificação clara e precisa de uma solução de forma independente de linguagem de programação
        % \item Define several commands to realize calculations within a \LaTeX
        % \begin{itemize}
        %     \item Um
        %     \item Dois
        % \end{itemize}
        % Next
        \item document. The calculator package introduces
        \item several new instructions that allow you to calculate with integer and real numbers using
        \pause
        \item [$\equiv$]\textbf{Ci}\textit{nco}!
    \end{mybox}

    \bigskip
    Or not\ldots
\end{frame}

\end{document}