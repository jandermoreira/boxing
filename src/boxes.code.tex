% boxes.code

% Variables
\NewDocumentCommand{\boxing@Colums}{}{}
\NewDocumentCommand{\boxing@EqualHeight}{}{}
\newif\ifboxing@Continued

\ExplSyntaxOn
\NewDocumentCommand{\RenewFullyExpandedDocumentCommand} { m m }{
    \exp_args:NNnV \RenewDocumentCommand #1 { } { #2 }
}
\ExplSyntaxOff

%! formatter = off
\pgfkeys{
    % /boxing/.is family,
    /boxing/.cd,
    % number of columns
    columns/.is choice,
    columns/auto/.code = {%
        \RenewDocumentCommand{\boxing@Colums}{}{\thenumgathereditems}%
    },
    columns/.unknown/.code = {%
        \RenewFullyExpandedDocumentCommand{\boxing@Colums}{\pgfkeyscurrentname}%
    },
    equal height/.is choice,
    equal height/none/.code = {%
        \RenewDocumentCommand{\boxing@EqualHeight}{}{none}%
    },
    equal height/rows/.code = {%
        \RenewDocumentCommand{\boxing@EqualHeight}{}{rows}%
    },
    equal height/all/.code = {%
        \RenewDocumentCommand{\boxing@EqualHeight}{}{all}%
    },
    % continue numbering?
    continued/.is if = boxing@Continued,
}
%! formatter = on

\NewDocumentCommand{\boxingset}{ >{\TrimSpaces} m }{%
    \pgfkeys{
        /boxing/.cd,
        #1
    }
}

% Default settings
\boxingset{
    columns = auto,
    equal height = all,
    continued = false,
}


% Tikz
\RequirePackage{tikz}
\tikzset{
    picture box/.style = {
        baseline = (text.base),
    },
    every text node/.style = {
        draw = \CurrentBGColor!70!black, ultra thick,
        ultra thick,
        text = \CurrentFGColor,
        font = \footnotesize,
        text width = \BoxWidth,
        inner xsep = 0.75em,
        inner ysep = 0.5em,
        shade,
        right color = \CurrentBGColor!70,
        left color = \CurrentBGColor,
        % bottom color = \CurrentBGColor!50,
    },
}

\NewDocumentCommand{\BoxBody}{}{}
\NewDocumentCommand{\HBox}{m}{%
    \UsePalette{palette_i}%
    \RenewDocumentCommand{\BoxBody}{}{\trim@spaces{#1}}%
    \begingroup%
    \ifUseBeamer%
    \setbeamerfont{itemize/enumerate body}{size = \footnotesize}%
    \setbeamercolor{itemize/enumerate body}{fg = \CurrentFGColor}%
    \setbeamercolor{structure}{fg = \CurrentFGColor}%
    \fi%
    \begin{tikzpicture}[picture box]
        \node[every text node] (text) {\raggedright\BoxBody\strut};
    \end{tikzpicture}%
    \endgroup%
    \StepPalette{palette_i}%
    \ifnum\thecurrentitemnumber=\thenumgathereditems\else%
    \hfill%
    \fi%
}

% Environments
\RequirePackage{getitems}

\NewEnviron{horizontalbox}[1][]{%
    \expandafter\gatheritems\expandafter{\BODY}%
    \ResetPalette{palette_i}%
    \setlength{\BoxWidth}{
        \dimexpr \textwidth/\thenumgathereditems - 1.5em -
        \BoxHSpacing
    }%
    \loopthroughitemswithcommand{\HBox}%
}

\RequirePackage{ragged2e}

\RequirePackage{tcolorbox}
\newif\ifUseIcon
\UseIconfalse
\tcbuselibrary{raster, skins}
\usetikzlibrary{calc}
\usetikzlibrary{overlay-beamer-styles}
\NewDocumentCommand{\TCBBox}{ m m +m }{%
    \visible<#1>{%
        \begin{tcolorbox}[
        left = 0pt,
        right = 0pt,
        bottom = 0pt,
        coltext = \CurrentFGColor,
        colback = \CurrentBGColor,
        colframe = \CurrentBGColor!70!black,
        toprule = 0.1em,
        bottomrule = 0.1em,
        leftrule = 0.1em,
        rightrule = 0.1em,
        enhanced,
        enlarge left by = 1em,
        enlarge top by = 1em,
        interior style = {
            top color = \CurrentBGColor,
            bottom color = \CurrentBGColor!60!white,
        },
        overlay = {
            \coordinate (North West) at ($(frame.north west) + (0.05em, -0.05em)$);
            \path [fill = \CurrentBGColor!90!black] (North West) circle (1em);
            \path [thick, draw = \CurrentBGColor!30!white] (North West) circle (0.9em);
            \path [draw = \CurrentBGColor!70!black, line width = 0.1em, ] ($(frame.north west) + (1.15em, -0.05em)$) -- ++(-0.1em, 0) arc (0:270:1em) -- ++(0, -0.1em);
            \node[font = \itshape, align = center, text = \CurrentFGColor] at ($(frame.north west) + (0.05em, -0.05em)$) {\ifUseIcon\includegraphics[width = 1.4em, height = 1.4em]{icons/#2}\else\scalebox{1.44}{#2}\fi};
        },
        ]
        \RaggedRight#3
        \end{tcolorbox}%
    }%
}
% \ExplSyntaxOn
% \cs_new:Npn \is_empty:N #1 {
%     \tl_count:N #1
% }
% \NewExpandableDocumentCommand{\IsEmpty}{ m }{
%     \is_empty:N { #1 }
% }
% \ExplSyntaxOff

\NewDocumentCommand{\TCB}{ +m }{%
    \UsePalette{Colored Coolers}%
    \ProcessItemArguments{\l@overlay}{\l@option}{\l@text}{#1}%
    %! formatter = off
    \ifx\l@overlay\empty%
        \RenewExpandableDocumentCommand{\l@overlay}{}{1-}%
    \fi
    \ifx\l@option\empty%
        \UseIconfalse%
        \RenewExpandableDocumentCommand{\l@option}{}{\theCurrentItemNumber}%
    \else
        \UseIcontrue%
    \fi%
    \ifUseBeamer%
        \setbeamerfont{itemize/enumerate body}{size = \footnotesize}%
        \setbeamercolor{itemize/enumerate body}{fg = \CurrentFGColor}%
        \setbeamercolor{structure}{fg = \CurrentFGColor}%
    \fi%
    %! formatter = on
    \TCBBox{\l@overlay}{\l@option}{\l@text}%
    \StepPalette{Colored Coolers}%
    \stepcounter{CurrentItemNumber}%
}

\newcounter{CurrentItemNumber}
\NewDocumentEnvironment{mybox}{ O{} +b }{%
    \begingroup%
    \gatheritems{#2}%
    %! formatter = off
    \boxingset{continued = false}
    \pgfkeys{%
        /boxing/.cd,
        #1
    }%
    \ifboxing@Continued\else
        \ResetPalette{Colored Coolers}%
        \setcounter{CurrentItemNumber}{1}%
    \fi
    %! formatter = on
    \begin{tcbraster}[
        raster columns = \boxing@Colums,
        raster halign = center,
        raster right skip = \dimexpr 1em * \kvtcb@raster@columns \relax,
        nobeforeafter,
        raster column skip = 0em,
        /utils/exec = {%
        \pgfkeysalsofrom{/tcb/raster equal height = \boxing@EqualHeight}%
        },
    ]
        \loopthroughitemswithcommand{\TCB}%
        }{%
    \end{tcbraster}%
    \endgroup%
}

\ExplSyntaxOn
\cs_new:Npn \gather_beamer_item_options:NNNn #1#2#3#4 {
    \tl_clear_new:N \g_item_overlay_spec_tl
    \tl_clear_new:N \g_item_option_spec_tl
    \tl_clear_new:N \g_item_text_tl
    \regex_extract_once:nnNTF { \A<(.*?)>\s*\[(.*?)\](.*) } { #4 } \g_item_structure_seq { % <> and []
        \tl_set:Nn \g_item_overlay_spec_tl { \seq_item:Nn \g_item_structure_seq { 2 } }
        \tl_set:Nn \g_item_option_spec_tl { \seq_item:Nn \g_item_structure_seq { 3 } }
        \tl_set:Nn \g_item_text_tl { \seq_item:Nn \g_item_structure_seq { 4 } }
    } { % [] and <>
        \regex_extract_once:nnNTF { \A\[(.*?)\]\s*<(.*?)>(.*) } { #4 } \g_item_structure_seq {
            \tl_gset:Nn \g_item_overlay_spec_tl { \seq_item:Nn \g_item_structure_seq { 3 } }
            \tl_gset:Nn \g_item_option_spec_tl { \seq_item:Nn \g_item_structure_seq { 2 } }
            \tl_gset:Nn \g_item_text_tl { \seq_item:Nn \g_item_structure_seq { 4 } }
        } { % only <>
            \regex_extract_once:nnNTF { \A<(.*?)>(.*) } { #4 } \g_item_structure_seq {
                \tl_gset:Nn \g_item_overlay_spec_tl { \seq_item:Nn \g_item_structure_seq { 2 } }
                % \tl_gset:Nn \g_item_option_spec_tl { \empty }
                \tl_gset:Nn \g_item_text_tl { \seq_item:Nn \g_item_structure_seq { 3 } }
            } { % only []
                \regex_extract_once:nnNTF { \A\[(.*?)\](.*) } { #4 } \g_item_structure_seq {
                % \tl_gset:Nn \g_item_overlay_spec_tl { \empty }
                    \tl_gset:Nn \g_item_option_spec_tl { \seq_item:Nn \g_item_structure_seq { 2 } }
                    \tl_gset:Nn \g_item_text_tl { \seq_item:Nn \g_item_structure_seq { 3 } }
                } { % only text
                % \tl_gset:Nn \g_item_overlay_spec_tl { \empty }
                % \tl_gset:Nn \g_item_option_spec_tl { \empty }
                    \tl_gset:Nn \g_item_text_tl { #4 }
                }
            }
        }
    }
    \exp_args:NNx \tl_set:Nn #1 { \g_item_overlay_spec_tl }
    \exp_args:NNV \tl_set:Nn #2 { \g_item_option_spec_tl }
    \exp_args:NNV \tl_set:Nn #3 { \g_item_text_tl }
}

\NewDocumentCommand{\ProcessItemArguments}{ m m m +m }{
    \gather_beamer_item_options:NNNn { #1 } { #2 } { #3 } { #4 }
}
\ExplSyntaxOff

\endinput